%!TEX TS-program = xelatex
%!TEX encoding = UTF-8 Unicode
% Awesome CV LaTeX Template for CV/Resume
%
% This template has been downloaded from:
% https://github.com/posquit0/Awesome-CV
%
% Author:
% Claud D. Park <posquit0.bj@gmail.com>
% http://www.posquit0.com
%
%
% Adapted to be an Rmarkdown template by Mitchell O'Hara-Wild
% 23 November 2018
%
% Template license:
% CC BY-SA 4.0 (https://creativecommons.org/licenses/by-sa/4.0/)
%
%-------------------------------------------------------------------------------
% CONFIGURATIONS
%-------------------------------------------------------------------------------
% A4 paper size by default, use 'letterpaper' for US letter
\documentclass[11pt, a4paper]{awesome-cv}

% Configure page margins with geometry
\geometry{left=1.4cm, top=.8cm, right=1.4cm, bottom=1.8cm, footskip=.5cm}

% Specify the location of the included fonts
\fontdir[fonts/]

% Color for highlights
% Awesome Colors: awesome-emerald, awesome-skyblue, awesome-red, awesome-pink, awesome-orange
%                 awesome-nephritis, awesome-concrete, awesome-darknight

\definecolor{awesome}{HTML}{414141}

% Colors for text
% Uncomment if you would like to specify your own color
% \definecolor{darktext}{HTML}{414141}
% \definecolor{text}{HTML}{333333}
% \definecolor{graytext}{HTML}{5D5D5D}
% \definecolor{lighttext}{HTML}{999999}

% Set false if you don't want to highlight section with awesome color
\setbool{acvSectionColorHighlight}{true}

% If you would like to change the social information separator from a pipe (|) to something else
\renewcommand{\acvHeaderSocialSep}{\quad\textbar\quad}

\def\endfirstpage{\newpage}

%-------------------------------------------------------------------------------
%	PERSONAL INFORMATION
%	Comment any of the lines below if they are not required
%-------------------------------------------------------------------------------
% Available options: circle|rectangle,edge/noedge,left/right

\name{Tom J}{Langbehn}

\position{Postdoctoral Research Fellow, Theoretical Ecology Group}
\address{Department of Biological Sciences, University of Bergen, Norway}

\email{\href{mailto:tom.langbehn@uib.no}{\nolinkurl{tom.langbehn@uib.no}}}
\homepage{bio.uib.no/te/tl/}
\orcid{0000-0003-1208-4793}
\googlescholar{upYmy0oAAAAJ}
\github{tomlangbehn}
\twitter{TomJasperL}

% \gitlab{gitlab-id}
% \stackoverflow{SO-id}{SO-name}
% \skype{skype-id}
% \reddit{reddit-id}

\quote{I am a marine biologist, I use models that couple the environment,
ecology \& evolution to understand how processes at the individual level
drive ecosystems}

\usepackage{booktabs}

\providecommand{\tightlist}{%
	\setlength{\itemsep}{0pt}\setlength{\parskip}{0pt}}

%------------------------------------------------------------------------------


\usepackage{fancyhdr}
\pagestyle{fancy}
\fancyhf{}
\fancyhead[R]{\thepage}

% Pandoc CSL macros
\newlength{\cslhangindent}
\setlength{\cslhangindent}{1.5em}
\newlength{\csllabelwidth}
\setlength{\csllabelwidth}{3em}
\newenvironment{CSLReferences}[3] % #1 hanging-ident, #2 entry spacing
 {% don't indent paragraphs
  \setlength{\parindent}{0pt}
  % turn on hanging indent if param 1 is 1
  \ifodd #1 \everypar{\setlength{\hangindent}{\cslhangindent}}\ignorespaces\fi
  % set entry spacing
  \ifnum #2 > 0
  \setlength{\parskip}{#2\baselineskip}
  \fi
 }%
 {}
\usepackage{calc}
\newcommand{\CSLBlock}[1]{#1\hfill\break}
\newcommand{\CSLLeftMargin}[1]{\parbox[t]{\csllabelwidth}{#1}}
\newcommand{\CSLRightInline}[1]{\parbox[t]{\linewidth - \csllabelwidth}{#1}}
\newcommand{\CSLIndent}[1]{\hspace{\cslhangindent}#1}

\begin{document}

% Print the header with above personal informations
% Give optional argument to change alignment(C: center, L: left, R: right)
\makecvheader

% Print the footer with 3 arguments(<left>, <center>, <right>)
% Leave any of these blank if they are not needed
% 2019-02-14 Chris Umphlett - add flexibility to the document name in footer, rather than have it be static Curriculum Vitae
\makecvfooter
  {December 19, 2020}
    {Tom J Langbehn~~~·~~~Curriculum Vitae}
  {\thepage}


%-------------------------------------------------------------------------------
%	CV/RESUME CONTENT
%	Each section is imported separately, open each file in turn to modify content
%------------------------------------------------------------------------------



\hypertarget{education}{%
\section{Education}\label{education}}

\begin{cventries}
    \cventry{Department of Biological Sciences, University of Bergen}{Research Fellow / Doctoral student (PhD)}{Bergen, Norway}{Feb 2016 - Mar 2019}{\begin{cvitems}
\item Dissertation title:  ``Light and visual foraging in the pelagic: Opportunities and constraints along gradients of seasonality''
\item Committee members: Prof. PhD Christian Jørgensen (University of Bergen), Prof. PhD Øyvind Fiksen (University of Bergen) and Associate Prof. PhD Øystein Varpe (University Centre in Svalbard).
\item Opponents: Prof. PhD Michael T. Burrows (Scottish Association for Marine Science, UK), PhD Xabier Irigoien (AZTI, Spain)
\end{cvitems}}
    \cventry{Faculty 2 Biology/Chemistry, University of Bremen}{Graduate Programme Marine Biology (M.Sc.)}{Bremen, Germany}{Oct 2013 - Oct 2015}{\begin{cvitems}
\item Thesis title: ``Feeding success in an extreme light environment: modelling seasonal prey encounter of Arctic fish''
\item Committee members: Prof. Dr. Wilhelm Hagen (University of Bremen) and Associate Prof. PhD Øystein Varpe (University Centre in Svalbard)
\item Final grade: Excellent
\end{cvitems}}
    \cventry{Faculty 5 Nature and Engineering, University of Applied Sciences Bremen}{International Degree Course in Environmental and Industrial Biology (B.Sc.)}{Bremen, Germany}{Sep 2009 - Feb 2013}{\begin{cvitems}
\item Thesis title: ``Morphological diversity in parr of Atlantic salmon (Salmo salar L.) in Iceland''
\item Committee members: Prof. Dr. Heiko Brunken (University of Applied Sciences Bremen) and Prof. PhD Guðrún Marteinsdóttir (University of Iceland)
\item Final grade: Excellent
\end{cvitems}}
\end{cventries}

\hypertarget{research-experience}{%
\section{Research Experience}\label{research-experience}}

\begin{cventries}
    \cventry{Theoretical Ecology Group, Department of Biological Sciences, University of Bergen}{Postdoctoral Research Fellow}{Bergen, Norway}{Sep 2019 - Present}{\begin{cvitems}
\item Project: ``The fundamental role of mesopelagic fishes for the structure and change of Northeast Atlantic marine ecosystems'', funded by the Research Council of Norway, \#294819
\end{cvitems}}
    \cventry{Arven etter Nansen (Nansen legacy), University Centre in Svalbard}{Researcher}{Longyearbyen, Svalbard, Norway}{Feb 2019 - Mar 2020}{\begin{cvitems}
\item Project: ``The pelagic riskscape and consequences for zooplankton size along gradients of light and sea-ice in the Barents Sea''
\end{cvitems}}
    \cventry{Thuenen Institute of Baltic Sea Fisheries}{Research Assistant}{Rostock, Germany}{Jul 2014 - Nov 2014}{\begin{cvitems}
\item Responsibility: net and acoustic sampling during the Rügen Herring larvae survey
\end{cvitems}}
    \cventry{BUND, Friends of the Earth Germany}{Freelance Ecological Consultant}{Bremen, Germany}{2012 -2014}{\begin{cvitems}
\item Contracted to develop and evaluate nature friendly methods of ditch clearance in FFH sites of high importance for the aquatic fauna
\end{cvitems}}
    \cventry{EcoSURV.Hein - Angewandte Fisch- \& Gewässerökologie}{Freelance Ecological Consultant}{Bremen, Germany}{2013}{\begin{cvitems}
\item Contracted to assist in electro fishing surveys
\end{cvitems}}
    \cventry{ÖKOLOGIS Umweltanalyse + Landschaftsplanung GmbH}{Freelance Ecological Consultant}{Bremen, Germany}{2013}{\begin{cvitems}
\item Contracted to map selected breeding birds in the FFH-site Werderland
\end{cvitems}}
    \cventry{Bioconsult SH GmbH \& Co. KG}{Freelance Ecological Consultant}{Husum, Germany}{2009 -2011}{\begin{cvitems}
\item Contracted to assist in offshore maintainance of harbour porpoise detectors, seabird at sea counts, and radar image assessment for quantitative migratory bird counts used in environmental impact assessments
\end{cvitems}}
    \cventry{Schutzstation-Wattenmeer e.V., Germany}{National Park Warden}{Hörnum, Sylt, Germany}{Oct 2008 - Jun 2009}{\begin{cvitems}
\item In fullfilment of  the German Civilian Service
\end{cvitems}}
\end{cventries}

\hypertarget{working-experience}{%
\section{Working Experience}\label{working-experience}}

\begin{cventries}
    \cventry{Freelance tour guide}{Svalbard Wildlife Expeditions AS}{Longyearbyen, Svalbard, Norway}{May - Jul 2015}{\begin{cvitems}
\item Guided tours on the history and wildlife of Svalbard
\end{cvitems}}
\end{cventries}

\hypertarget{academic-mobility}{%
\section{Academic mobility}\label{academic-mobility}}

*only stays longer than two month are listed

\begin{cventries}
    \cventry{Centre for Ocean Life, National Institute of Aquatic Resources, Technical University of Denmark}{PhD research secondment}{Copenhagen, Denmark}{Nov - Oct 2017}{\begin{cvitems}
\item 2 month
\item Project: ``Can a food systems approach redefine ecosystem-based fisheries management?''
\item Visiting Ken H. Andersen
\end{cvitems}}
    \cventry{Department of Artic Biology, University Centre in Svalbard}{Guest Master student}{Longyearbyen, Svalbard, Norway}{Mar - Oct 2015}{\begin{cvitems}
\item 8 month
\item Project: ``Feeding success in an extreme light environment: modelling seasonal prey encounter of Arctic fish''
\item Visiting Øystein Varpe
\end{cvitems}}
    \cventry{Marine Academic Research in Iceland (MARICE), Institute of Biology, University of Iceland}{Undergraduate research semester abroad}{Reykjavík, Iceland}{Jan - May 2012}{\begin{cvitems}
\item 5 month
\item Project: ``Impact of climate change on processes that influence survival of Atlantic cod eggs and larvae in Icelandic waters''
\item Visiting Guðrún Marteinsdóttir
\end{cvitems}}
    \cventry{Institute of Biology, University of Iceland}{Graduate Marine and Fisheries Sciences Summer Programme}{Reykjavík, Iceland}{May - Jul 2012}{\begin{cvitems}
\item 3 month
\item Courses:  ``Data Analysis for Scientists using R'', ``Fisheries Ecology'' and ``Marine Mammals''
\item Final grade:  First class with distinction
\end{cvitems}}
    \cventry{Department of Artic Biology, University Centre in Svalbard}{Graduate Course Ecosystem-based Management of Arctic Marine Systems}{Longyearbyen, Svalbard, Norway}{Aug - Sep 2014}{\begin{cvitems}
\item 2 month
\item Final grade: Excellent
\end{cvitems}}
    \cventry{Inland Norway University of Applied Sciences}{Undergraduate study semester abroad}{Evenstad, Norway}{Aug - Dec 2011}{\begin{cvitems}
\item 5 month
\item International Degree Course Ecology and Conservation
\item Courses:  ``Evolution'', ``Conservation Biology'', ``Population Dynamics'' and ``Practice in Wildlife Research''
\item Final grade: Excellent
\end{cvitems}}
\end{cventries}

\hypertarget{awards-and-honors}{%
\section{Awards and Honors}\label{awards-and-honors}}

\begin{cventries}
    \cventry{ICES Annual Science Conference 2019}{Best Presentation Award}{Hamburg, Germany}{2019}{\vspace{-4mm}}
    \cventry{IMBER ClimEco5 Summer School}{Student Choice and Special Mention Poster Award}{Natal, Brazil}{2016}{\vspace{-4mm}}
    \cventry{YOUMARES 7 conference, German Society for Marine Research}{Best Poster Award}{Hamburg, Germany}{2016}{\vspace{-4mm}}
    \cventry{MARUM Center for Marine Environmental Science}{MARUM Research Award for Marine Science}{Bremen, Germany}{2015}{\begin{cvitems}
\item awarded in recognition of an outstanding Masters
\end{cvitems}}
    \cventry{YOUMARES 6 conference, German Society for Marine Research}{Best Speaker Award}{Bremen, Germany}{2015}{\vspace{-4mm}}
    \cventry{Fisheries Ecology Summer School, University of Iceland}{Best Student Poster}{Reykjavík, Iceland}{2011}{\vspace{-4mm}}
\end{cventries}

\end{document}
