%!TEX TS-program = xelatex
%!TEX encoding = UTF-8 Unicode
% Awesome CV LaTeX Template for CV/Resume
%
% This template has been downloaded from:
% https://github.com/posquit0/Awesome-CV
%
% Author:
% Claud D. Park <posquit0.bj@gmail.com>
% http://www.posquit0.com
%
%
% Adapted to be an Rmarkdown template by Mitchell O'Hara-Wild
% 23 November 2018
%
% Template license:
% CC BY-SA 4.0 (https://creativecommons.org/licenses/by-sa/4.0/)
%
%-------------------------------------------------------------------------------
% CONFIGURATIONS
%-------------------------------------------------------------------------------
% A4 paper size by default, use 'letterpaper' for US letter
\documentclass[11pt, a4paper]{awesome-cv}

% Configure page margins with geometry
\geometry{left=1.4cm, top=.8cm, right=1.4cm, bottom=1.8cm, footskip=.5cm}

% Specify the location of the included fonts
\fontdir[fonts/]

% Color for highlights
% Awesome Colors: awesome-emerald, awesome-skyblue, awesome-red, awesome-pink, awesome-orange
%                 awesome-nephritis, awesome-concrete, awesome-darknight

\definecolor{awesome}{HTML}{414141}

% Colors for text
% Uncomment if you would like to specify your own color
% \definecolor{darktext}{HTML}{414141}
% \definecolor{text}{HTML}{333333}
% \definecolor{graytext}{HTML}{5D5D5D}
% \definecolor{lighttext}{HTML}{999999}

% Set false if you don't want to highlight section with awesome color
\setbool{acvSectionColorHighlight}{true}

% If you would like to change the social information separator from a pipe (|) to something else
\renewcommand{\acvHeaderSocialSep}{\quad\textbar\quad}

\def\endfirstpage{\newpage}

%-------------------------------------------------------------------------------
%	PERSONAL INFORMATION
%	Comment any of the lines below if they are not required
%-------------------------------------------------------------------------------
% Available options: circle|rectangle,edge/noedge,left/right

\name{Tom J}{Langbehn}

\position{Postdoctoral Research Fellow, Theoretical Ecology Group}
\address{Department of Biological Sciences, University of Bergen, Norway}

\email{\href{mailto:tom.langbehn@uib.no}{\nolinkurl{tom.langbehn@uib.no}}}
\homepage{bio.uib.no/te/tl/}
\orcid{0000-0003-1208-4793}
\googlescholar{upYmy0oAAAAJ}
\github{tomlangbehn}
\twitter{TomJasperL}

% \gitlab{gitlab-id}
% \stackoverflow{SO-id}{SO-name}
% \skype{skype-id}
% \reddit{reddit-id}

\quote{I am a marine biologist with a keen interested in evolutionary ecology
and global change biology, i.e., how climate change and human activities
such as fisheries shape biological interactions and processes across
various levels of organisation. I aim to solve real-life problems by
identifying and understanding how processes at the individual level
drive ecosystems. In my work, I combine mechanistic modelling often
informed by theory with observation from the field. I enjoy thinking
broadly and across disciplines and have a particular fascination for
polar ecosystems and the ocean twilight zone.}

\usepackage{booktabs}

\providecommand{\tightlist}{%
	\setlength{\itemsep}{0pt}\setlength{\parskip}{0pt}}

%------------------------------------------------------------------------------


\usepackage{fancyhdr}
\pagestyle{fancy}
\fancyhf{}
\fancyhead[R]{\thepage}

% Pandoc CSL macros
\newlength{\cslhangindent}
\setlength{\cslhangindent}{1.5em}
\newlength{\csllabelwidth}
\setlength{\csllabelwidth}{3em}
\newenvironment{CSLReferences}[3] % #1 hanging-ident, #2 entry spacing
 {% don't indent paragraphs
  \setlength{\parindent}{0pt}
  % turn on hanging indent if param 1 is 1
  \ifodd #1 \everypar{\setlength{\hangindent}{\cslhangindent}}\ignorespaces\fi
  % set entry spacing
  \ifnum #2 > 0
  \setlength{\parskip}{#2\baselineskip}
  \fi
 }%
 {}
\usepackage{calc}
\newcommand{\CSLBlock}[1]{#1\hfill\break}
\newcommand{\CSLLeftMargin}[1]{\parbox[t]{\csllabelwidth}{#1}}
\newcommand{\CSLRightInline}[1]{\parbox[t]{\linewidth - \csllabelwidth}{#1}}
\newcommand{\CSLIndent}[1]{\hspace{\cslhangindent}#1}

\begin{document}

% Print the header with above personal informations
% Give optional argument to change alignment(C: center, L: left, R: right)
\makecvheader

% Print the footer with 3 arguments(<left>, <center>, <right>)
% Leave any of these blank if they are not needed
% 2019-02-14 Chris Umphlett - add flexibility to the document name in footer, rather than have it be static Curriculum Vitae
\makecvfooter
  {December 21, 2020}
    {Tom J Langbehn~~~·~~~Curriculum Vitae}
  {\thepage}


%-------------------------------------------------------------------------------
%	CV/RESUME CONTENT
%	Each section is imported separately, open each file in turn to modify content
%------------------------------------------------------------------------------



\hypertarget{education}{%
\section{Education}\label{education}}

\begin{cventries}
    \cventry{Department of Biological Sciences, University of Bergen}{Research Fellow / Doctoral student (PhD)}{Bergen, Norway}{Feb 2016 - Mar 2019}{\begin{cvitems}
\item Dissertation title:  ``Light and visual foraging in the pelagic: Opportunities and constraints along gradients of seasonality''
\item Committee members: Prof. PhD Christian Jørgensen (University of Bergen), Prof. PhD Øyvind Fiksen (University of Bergen) and Associate Prof. PhD Øystein Varpe (University Centre in Svalbard).
\item Opponents: Prof. PhD Michael T. Burrows (Scottish Association for Marine Science, UK), PhD Xabier Irigoien (AZTI, Spain)
\item As part of the  European Training Network on``Marine Management and Ecosystem Dynamics under Climate Change - MARmaED''
\end{cvitems}}
    \cventry{University of Bergen}{Bergen Summer research School: Water, Climate \& Society}{Bergen, Norway}{Aug 2016}{\vspace{-4mm}}
    \cventry{IMBeR}{ClimEco5 Summer School}{Natal, Brazil}{Jun 2016}{\begin{cvitems}
\item Topic: ``Towards more resilient oceans: Predicting and projecting future changes in the ocean and their impacts on human societies''
\end{cvitems}}
    \cventry{Faculty 2 Biology/Chemistry, University of Bremen}{Graduate Programme Marine Biology (M.Sc.)}{Bremen, Germany}{Oct 2013 - Oct 2015}{\begin{cvitems}
\item Thesis title: ``Feeding success in an extreme light environment: modelling seasonal prey encounter of Arctic fish''
\item Committee members: Prof. Dr. Wilhelm Hagen (University of Bremen) and Associate Prof. PhD Øystein Varpe (University Centre in Svalbard)
\item Final grade: Excellent
\end{cvitems}}
    \cventry{Faculty 5 Nature and Engineering, University of Applied Sciences Bremen}{International Degree Course in Environmental and Industrial Biology (B.Sc.)}{Bremen, Germany}{Sep 2009 - Feb 2013}{\begin{cvitems}
\item Thesis title: ``Morphological diversity in parr of Atlantic salmon (Salmo salar L.) in Iceland''
\item Committee members: Prof. Dr. Heiko Brunken (University of Applied Sciences Bremen) and Prof. PhD Guðrún Marteinsdóttir (University of Iceland)
\item Final grade: Excellent
\end{cvitems}}
\end{cventries}

\hypertarget{research-experience}{%
\section{Research experience}\label{research-experience}}

\begin{cventries}
    \cventry{Theoretical Ecology Group, Department of Biological Sciences, University of Bergen}{Postdoctoral Research Fellow}{Bergen, Norway}{Sep 2019 - Present}{\begin{cvitems}
\item Project: ``The fundamental role of mesopelagic fishes for the structure and change of Northeast Atlantic marine ecosystems'', funded by the Research Council of Norway, \#294819
\end{cvitems}}
    \cventry{Arven etter Nansen (Nansen legacy), University Centre in Svalbard}{Researcher}{Longyearbyen, Svalbard, Norway}{Feb 2019 - Mar 2020}{\begin{cvitems}
\item Project: ``The pelagic riskscape and consequences for zooplankton size along gradients of light and sea-ice in the Barents Sea''
\end{cvitems}}
    \cventry{Thuenen Institute of Baltic Sea Fisheries}{Research Assistant}{Rostock, Germany}{Jul 2014 - Nov 2014}{\begin{cvitems}
\item Assisting during the Rügen Herring larvae survey
\end{cvitems}}
\end{cventries}

\hypertarget{working-experience}{%
\section{Working Experience}\label{working-experience}}

\begin{cventries}
    \cventry{Svalbard Wildlife Expeditions AS}{Freelance tour guide}{Longyearbyen, Svalbard, Norway}{May - Jul 2015}{\begin{cvitems}
\item Guided tours on the history and wildlife of Svalbard
\end{cvitems}}
    \cventry{BUND, Friends of the Earth Germany}{Freelance Ecological Consultant}{Bremen, Germany}{2012 -2014}{\begin{cvitems}
\item Contracted to develop and evaluate nature friendly methods of ditch clearance in FFH sites of high importance for the aquatic fauna
\end{cvitems}}
    \cventry{EcoSURV.Hein - Angewandte Fisch- \& Gewaesseroekologie}{Freelance Ecological Consultant}{Bremen, Germany}{2013}{\begin{cvitems}
\item Contracted to assist in electro fishing surveys
\end{cvitems}}
    \cventry{ÖKOLOGIS Umweltanalyse + Landschaftsplanung GmbH}{Freelance Ecological Consultant}{Bremen, Germany}{2013}{\begin{cvitems}
\item Contracted to map selected breeding birds in the FFH-site Werderland
\end{cvitems}}
    \cventry{Bioconsult SH GmbH \& Co. KG}{Freelance Ecological Consultant}{Husum, Germany}{2009 -2011}{\begin{cvitems}
\item Contracted to assist in offshore maintainance of harbour porpoise detectors, seabird at sea counts, and radar image assessment for quantitative migratory bird counts used in environmental impact assessments
\end{cvitems}}
    \cventry{Schutzstation-Wattenmeer e.V., Germany}{National Park Warden}{Hörnum, Sylt, Germany}{Oct 2008 - Jun 2009}{\begin{cvitems}
\item In fullfilment of  the German Civilian Service
\item Responsiblilities included to oversee several scientific monitoring programs (e.g migratory bird counts and OSPAR litter monitoring), outreach and environmental education, as well as local coordination with nature conservation authorities
\item Civilian service was extended by two internships periods, Aug 2008 - Sep 2008 and Jun 2009 - Aug 2009
\end{cvitems}}
\end{cventries}

\hypertarget{academic-mobility}{%
\section{Academic mobility}\label{academic-mobility}}

only stays longer than two month are listed

\begin{cventries}
    \cventry{Centre for Ocean Life, National Institute of Aquatic Resources, Technical University of Denmark}{PhD research secondment}{Copenhagen, Denmark}{Nov - Oct 2017}{\begin{cvitems}
\item 2 month
\item Project: ``Can a food systems approach redefine ecosystem-based fisheries management?''
\item Visiting Ken H. Andersen
\end{cvitems}}
    \cventry{Department of Artic Biology, University Centre in Svalbard}{Guest Master student}{Longyearbyen, Svalbard, Norway}{Mar - Oct 2015}{\begin{cvitems}
\item 8 month
\item Project: ``Feeding success in an extreme light environment: modelling seasonal prey encounter of Arctic fish''
\item Visiting Øystein Varpe
\end{cvitems}}
    \cventry{Department of Artic Biology, University Centre in Svalbard}{Graduate Course Ecosystem-based Management of Arctic Marine Systems}{Longyearbyen, Svalbard, Norway}{Aug - Sep 2014}{\begin{cvitems}
\item 2 month
\item Final grade: Excellent
\end{cvitems}}
    \cventry{Institute of Biology, University of Iceland}{Graduate Marine and Fisheries Sciences Summer Programme}{Reykjavík, Iceland}{May - Jul 2012}{\begin{cvitems}
\item 3 month
\item Courses:  ``Data Analysis for Scientists using R'', ``Fisheries Ecology'' and ``Marine Mammals''
\item Final grade:  First class with distinction
\end{cvitems}}
    \cventry{Marine Academic Research in Iceland (MARICE), Institute of Biology, University of Iceland}{Undergraduate research semester abroad}{Reykjavík, Iceland}{Jan - May 2012}{\begin{cvitems}
\item 5 month
\item Project: ``Impact of climate change on processes that influence survival of Atlantic cod eggs and larvae in Icelandic waters''
\item Visiting Guðrún Marteinsdóttir
\end{cvitems}}
    \cventry{Inland Norway University of Applied Sciences}{Undergraduate study semester abroad}{Evenstad, Norway}{Aug - Dec 2011}{\begin{cvitems}
\item 5 month
\item International Degree Course Ecology and Conservation
\item Courses:  ``Evolution'', ``Conservation Biology'', ``Population Dynamics'' and ``Practice in Wildlife Research''
\item Final grade: Excellent
\end{cvitems}}
\end{cventries}

\hypertarget{awards-and-honors}{%
\section{Awards and Honors}\label{awards-and-honors}}

\begin{cventries}
    \cventry{ICES Annual Science Conference 2019}{Best Presentation Award}{Hamburg, Germany}{2019}{\vspace{-4mm}}
    \cventry{YOUMARES 7 conference, German Society for Marine Research}{Best Poster Award}{Hamburg, Germany}{2016}{\vspace{-4mm}}
    \cventry{IMBER ClimEco5 Summer School}{Student Choice and Special Mention Poster Award}{Natal, Brazil}{2016}{\vspace{-4mm}}
    \cventry{MARUM Center for Marine Environmental Science}{MARUM Research Award for Marine Science}{Bremen, Germany}{2015}{\begin{cvitems}
\item Awarded in recognition of an outstanding Masters
\end{cvitems}}
    \cventry{YOUMARES 6 conference, German Society for Marine Research}{Best Speaker Award}{Bremen, Germany}{2015}{\vspace{-4mm}}
    \cventry{Fisheries Ecology Summer School, University of Iceland}{Best Student Poster}{Reykjavík, Iceland}{2011}{\vspace{-4mm}}
\end{cventries}

\hypertarget{research-cruises}{%
\section{Research Cruises}\label{research-cruises}}

I have participated in \textbf{9 research cruises}, including
\textbf{one as a cruise leader}, and spent in total
\textbf{\textgreater{} 50 days at sea}, with some cruises lasting up to
two weeks. I have worked both on research and commercial fishing
vessels, in the North Sea, the Baltic Sea, Icelandic waters, Norwegian
coastal waters, the Barents Sea and Arctic waters up to the marginal ice
zone.

\hypertarget{organisation-of-scientific-meetings}{%
\section{Organisation Of Scientific
Meetings}\label{organisation-of-scientific-meetings}}

\begin{cventries}
    \cventry{International Council for the Exploration of the Sea}{Theme session convenor, ICES Annual Science Conference}{Copenhagen, Denmark}{2020}{\begin{cvitems}
\item Session title: Biomass, biodiversity, and ecosystem services in the mesopelagic zone
\item Co-conveners: Helena McMonagle (UW SAFS, USA), Peter H. Wiebe (WHOI, USA)
\item Due to COIVD-19 the  ICES ASC 2020 has been postponed to 2021
\end{cvitems}}
    \cventry{Norsk Havforskerforening}{Annual meeting of the Norwegian Association of Marine Scientists}{Bergen, Norway}{2017}{\begin{cvitems}
\item Part of the local organisers
\end{cvitems}}
    \cventry{German Society for Marine Research (DGM)}{Theme session convenor, YOUMARES 7}{Hamburg, Germany}{2016}{\begin{cvitems}
\item Session title: ``DEEP | DARK | COLD - frontiers in polar and deep-sea research''
\end{cvitems}}
\end{cventries}

\hypertarget{current-memberships}{%
\section{Current Memberships}\label{current-memberships}}

\begin{itemize}
\tightlist
\item
  since 2017 - Norwegian Association of Marine Scientists / Norsk
  Havforskerforening (NHF)
\item
  since 2015 - Association of Polar Early Career Scientists (APECS)
\end{itemize}

\hypertarget{teaching-experience}{%
\section{Teaching Experience}\label{teaching-experience}}

\hypertarget{university-courses}{%
\subsection{University courses}\label{university-courses}}

\begin{cventries}
    \cventry{University of Bergen}{Guest lecturer – BIO300A Academic writing}{Bergen Norway}{Nov 2020}{\begin{cvitems}
\item Title: ``Visual storytelling''
\end{cvitems}}
    \cventry{University of Bergen}{Guest lecturer –  SDG214 Life below water}{Bergen Norway}{May 2019}{\begin{cvitems}
\item Title: ``How to design better scientific figures  and posters''
\end{cvitems}}
    \cventry{University of Bergen}{Guest lecturer –  BIO241 Behavioural Ecology}{Bergen Norway}{May 2019}{\begin{cvitems}
\item Title: ``How to design better scientific figures  and posters''
\end{cvitems}}
    \cventry{University Centre in Svalbard}{Guest lecturer –  AB338 Life History Adaptations to Seasonality}{Longyearbyen, Svalbard, Norway}{Jun 2018}{\begin{cvitems}
\item Title: ``Climate-driven range shifts and life history adaptations to seasonality''
\end{cvitems}}
    \cventry{University Centre in Svalbard}{Teaching assistant – AB204 Arctic Ecology and Population Biology}{Longyearbyen, Svalbard, Norway}{Sep 2016}{\begin{cvitems}
\item Supervised the integrated Teach2Learn module as part of bioCEED a Centre of Excellence in Biology Education
\end{cvitems}}
    \cventry{University Centre in Svalbard}{Teaching assistant – AB202 Marine Arctic Biology}{Longyearbyen, Svalbard, Norway}{May 2015}{\begin{cvitems}
\item Responsible for seabird at sea census techniques and fisheries data sampling
\end{cvitems}}
    \cventry{University of Applied Sciences Bremen}{Teaching assistant – Math preparation course for BSc students}{Bremen, Germany}{Oct 2010}{\vspace{-4mm}}
\end{cventries}

\hypertarget{webinars}{%
\subsection{Webinars}\label{webinars}}

\begin{cventries}
    \cventry{DEEP: Norwegian Research School for Dynamics and Evolution of Earth and Planets}{Visual storytelling}{Virtual}{Dec 2020}{\vspace{-4mm}}
    \cventry{
bioCEED – Centre for Excellence in Biology Education}{Creating good scientific posters}{Virtual}{Nov 2020}{\vspace{-4mm}}
    \cventry{Nansen Legacy}{Creating good scientific posters}{Virtual}{Oct 2020}{\vspace{-4mm}}
\end{cventries}

\hypertarget{workshops}{%
\subsection{Workshops}\label{workshops}}

\begin{cventries}
    \cventry{Nansen Legacy}{Introduction to data wrangling with tidyverse}{Virtual}{Jan 2021}{\vspace{-4mm}}
\end{cventries}

\hypertarget{supervision}{%
\section{Supervision}\label{supervision}}

\begin{cventries}
    \cventry{Master thesis}{Astrid Holtan Fredriksen, MSc}{}{Aug 2019 - Jul 2021}{\begin{cvitems}
\item Project: ``Trophic position of myctophids in the Northeast Atlantic''
\end{cvitems}}
    \cventry{Master thesis}{Andéol Bourgouin, Msc}{}{Jan  2020 - Jul 2020}{\begin{cvitems}
\item Project: ``Exploring the emergent niche of Greater argentine (Argentina silus) along gradients of topography and light''
\end{cvitems}}
    \cventry{BIO299 Research Practice in Biology}{Cecilie Iden Nilsen, Bsc}{}{Sep 2019 - Apr 2020}{\begin{cvitems}
\item Project: ``Developing methods for lipid extraction in mesopelagic fish''
\end{cvitems}}
\end{cventries}

\hypertarget{publications}{%
\section{Publications}\label{publications}}

\hypertarget{publications-track-record}{%
\subsection{Publications track record}\label{publications-track-record}}

During my career I have published a \textbf{total number of 8
peer-reviewed papers}. Thereof \textbf{7 publications in peer-reviewed
scientific journals} and \textbf{1 in peer-reviewed conference
proceedings}. According to Google Scholar (as of 21 Dec 2020), these
have been \textbf{cited a total of 79 times}, thereof \textbf{45 times
within 2020}.

\hypertarget{articles-in-peer-reviewed-journals}{%
\subsection{Articles in Peer-reviewed
Journals}\label{articles-in-peer-reviewed-journals}}

*indicates equal author contributions

\begingroup
\setlength{\parindent}{-0.5in}
\setlength{\leftskip}{0.5in}

\hypertarget{refs_journals}{}
\leavevmode\hypertarget{ref-Butler2020}{}%
Butler, W. E., Guðmundsdóttir, L., Logemann, K.,
\textbf{Langbehn, T.J.}, \& Marteinsdóttir, G. (2020). Egg size and
density estimates for three gadoids in Icelandic waters and their
implications for the vertical distribution of eggs along a stratified
water column. \emph{Journal of Marine Systems}, \emph{204}(2019),
103290. \url{https://doi.org/10.1016/j.jmarsys.2019.103290}

\leavevmode\hypertarget{ref-Kaartvedt2019b}{}%
Kaartvedt, S., \textbf{Langbehn, T.J.}, \& Aksnes, D. L. (2019).
Enlightening the ocean's twilight zone. \emph{ICES Journal of Marine
Science}, \emph{76}(4), 803--812.
\url{https://doi.org/10.1093/icesjms/fsz010}

\leavevmode\hypertarget{ref-Langbehn2019a}{}%
\textbf{Langbehn, T.J.}, Aksnes, D., Kaartvedt, S., Fiksen, Ø., \&
Jørgensen, C. (2019). Light comfort zone in a mesopelagic fish emerges
from adaptive behaviour along a latitudinal gradient. \emph{Marine
Ecology Progress Series}, \emph{623}, 161--174.
\url{https://doi.org/10.3354/meps13024}

\leavevmode\hypertarget{ref-Sguotti}{}%
Sguotti, C., Otto, S. A., Frelat, R., \textbf{Langbehn, T.J.}, Ryberg,
M. P., Lindegren, M., \ldots{} Möllmann, C. (2019). Catastrophic
dynamics limit Atlantic cod recovery. \emph{Proceedings of the Royal
Society B: Biological Sciences}, \emph{286}(1898), 20182877.
\url{https://doi.org/10.1098/rspb.2018.2877}

\leavevmode\hypertarget{ref-Geoffroy2018}{}%
Geoffroy, M., Berge, J., Majaneva, S., Johnsen, G.,
\textbf{Langbehn, T.J.}, Cottier, F., \ldots{} Last, K. (2018).
Increased occurrence of the jellyfish *Periphylla periphylla* in the
European high Arctic. \emph{Polar Biology}, \emph{41}(12), 2615--2619.
\url{https://doi.org/10.1007/s00300-018-2368-4}

\leavevmode\hypertarget{ref-Langbehn2017}{}%
\textbf{Langbehn, T.J.}, \& Varpe, Ø. (2017). Sea-ice loss boosts visual
search: fish foraging and changing pelagic interactions in polar oceans.
\emph{Global Change Biology}, \emph{23}(12), 5318--5330.
\url{https://doi.org/10.1111/gcb.13797}. \emph{The Ecological Society of
America featured this article as a dispatch Melting sea ice may open
polar oceans to new predators in Frontiers in the Ecology and the
Environment}.

\leavevmode\hypertarget{ref-Katzenberger2013}{}%
Katzenberger, J., \textbf{Langbehn, T.J.}, \& Zacharias, D. (2013).
Erstnachweis von Hydrellia tarsata (Diptera: Ephydridae) für Bremen in
Blüten von Stratiotes aloides. \emph{Abhandlungen Des
Naturwissenschaftlichen Vereins Zu Bremen}, \emph{47}(1), 193--195.

\endgroup

\hypertarget{working-papers-under-revision-or-review}{%
\subsection{Working Papers under Revision or
Review}\label{working-papers-under-revision-or-review}}

\emph{in. prep.} articles are only listed when ready to submit

\begingroup
\setlength{\parindent}{-0.5in}
\setlength{\leftskip}{0.5in}

\hypertarget{refs_working_paper}{}
\leavevmode\hypertarget{ref-Geoffroy}{}%
Geoffroy, M., \textbf{Langbehn, T.J.}, Priou, P., Varpe, Ø., Johnsen,
G., LeBris, A., \ldots{} Berge, J. (in prep.). Pelagic fish and
zooplankton avoid artificial light from scientific instruments,
including red light. \emph{Tbd}.

\leavevmode\hypertarget{ref-Langbehnh}{}%
\textbf{Langbehn, T.J.}, Aksnes, D. L., Kaartvedt, S., Fiksen, Ø., \&
Jørgensen, C. (in prep.). Poleward distribution of mesopelagic fishes is
constrained by seasonality in light. \emph{Tbd}.

\leavevmode\hypertarget{ref-Ljungstrom}{}%
Ljungström*, G., \textbf{Langbehn*, T.J.}, \& Jørgensen, C. (under
revision). Light and energetics at seasonal extremes limit poleward
range shifts. \emph{Nature Climate Change}.

\endgroup

\hypertarget{articles-in-peer-reviewed-conference-proceedings}{%
\subsection{Articles in Peer-reviewed Conference
Proceedings}\label{articles-in-peer-reviewed-conference-proceedings}}

\begingroup
\setlength{\parindent}{-0.5in}
\setlength{\leftskip}{0.5in}

\hypertarget{refs_proceedings}{}
\leavevmode\hypertarget{ref-Langbehn2016c}{}%
\textbf{Langbehn, T.J.}, \& Schoenle, A. (2016). Deep \textbar{} Dark
\textbar{} Cold - Frontiers in Polar and Deep-Sea Research. In M. Bode,
C. Jessen, \& V. Golz (Eds.), \emph{Youmares 7 conference proceedings}
(pp. 59--68). German Society for Marine Research, Working Group on
Studies; Education, Deutsche Gesellschaft für Meeresforschung (DGM) e.V.

\endgroup

\hypertarget{talks}{%
\section{Talks}\label{talks}}

\begingroup
\setlength{\parindent}{-0.5in}
\setlength{\leftskip}{0.5in}

\hypertarget{refs_talks}{}
\leavevmode\hypertarget{ref-Langbehn12}{}%
\textbf{Langbehn, T.J.}, Aarflot, J. M., \& Varpe, Ø. (2021). \emph{The
pelagic riskscape and consequences for zooplankton size along gradients
of light and sea-ice in the barents sea}. ICES Annual Science Conference
2021, Copenhagen, Denmark. \emph{\textbf{accepted}}.

\leavevmode\hypertarget{ref-Langbehn13}{}%
\textbf{Langbehn, T.J.}, Aarflot, J. M., \& Varpe, Ø. (2021). \emph{The
pelagic riskscape and consequences for zooplankton size along gradients
of light and sea-ice in the barents sea}. ICES Annual Science Conference
2021, 6. - 9. Sep, Copenhagen, Denmark. \emph{\textbf{accepted}}.

\leavevmode\hypertarget{ref-Langbehn11}{}%
\textbf{Langbehn, T.J.}, Ljungström, G., \& Jørgensen, C. (2020).
\emph{Equatorwards shifts of marine fishes under climate change: A rule
rather than exception?} Ocean Sciences Meeting, 16.- 21. Feb, San Diego,
USA.

\leavevmode\hypertarget{ref-Langbehn6}{}%
\textbf{Langbehn, T.J.}, Aksnes, D. L., Kaartvedt, S., Fiksen, Ø., \&
Jørgensen, C. (2019). \emph{Light and the ecology of mesopelagic fish at
high latitudes}. Ocean Outlook 2019, 8. May, Bergen, Norway.
\emph{\textbf{Invited}}.

\leavevmode\hypertarget{ref-Langbehn9}{}%
\textbf{Langbehn, T.J.}, Aksnes, D. L., Kaartvedt, S., Fiksen, Ø., \&
Jørgensen, C. (2019). \emph{Poleward distribution of mesopelagic fish is
constrained by seasonality in light}. ICES Symposium Challenging The
Scientific Legacy Of Johan Hjort, 12.- 14. Jun, Bergen, Norway.

\leavevmode\hypertarget{ref-Langbehn5}{}%
\textbf{Langbehn, T.J.}, Jørgensen, C., Fiksen, Ø., Aksnes, D. L., \&
Kaartvedt, S. (2019). \emph{From light, to vision, to species
distribution}. Biodagen at the University of Bergen, 15. Mar, Bergen,
Norway. \emph{\textbf{Invited}}.

\leavevmode\hypertarget{ref-Langbehn8}{}%
\textbf{Langbehn, T.J.}, Jørgensen, C., Ljungström, G., Andersen, K. H.,
Aksnes, D. L., \& Enberg, K. (2019). \emph{Can a food systems approach
redefine ecosystem-based?} MARmaED Symposium, 25.- 26. May, Palma de
Mallorca, Spain.

\leavevmode\hypertarget{ref-Langbehn10}{}%
\textbf{Langbehn, T.J.}, \& Varpe, Ø. (2019). \emph{Poleward
distribution of mesopelagic fish is constrained by seasonality in
light}. BIARRITZ (Bridging International Activity and Related Research
Into the Twilight Zone) workshop, 22.- 26. Jun, Southampton, UK.
\emph{{[}Flash talk{]}}.

\leavevmode\hypertarget{ref-Langbehn7}{}%
\textbf{Langbehn, T.J.}, \& Varpe, Ø. (2019). \emph{Sea-ice loss boosts
visual search: Fish foraging and changing pelagic interactions in polar
oceans}. Polar Express, Polar Sciences Network, University of Bergen,
15. May, Bergen , Norway. \emph{{[}Flash talk{]}}.

\leavevmode\hypertarget{ref-Langbehn4}{}%
\textbf{Langbehn, T.J.}, Aksnes, D. L., Kaartvedt, S., Fiksen, Ø., \&
Jørgensen, C. (2018). \emph{The deadly midnight sun: Visual predators
and northern range limits in mesopelagic fish}. ICES annual science
conference, 24.- 27. Sep, Hamburg, Germany. \emph{\textbf{Best
Presentation Award}}.

\leavevmode\hypertarget{ref-Langbehn3}{}%
\textbf{Langbehn, T.J.}, \& Varpe, Ø. (2018). \emph{Sea-ice loss boosts
visual search: Fish foraging and changing pelagic interactions in polar
oceans}. ICES annual science conference, 24.- 27. Sep, Hamburg, Germany.

\leavevmode\hypertarget{ref-Langbehn2}{}%
\textbf{Langbehn, T.J.} (2017). \emph{Abundance and productivity of
mesopelagic fish}. Ocean Conference as part of the naming ceremony for
the new research vessel Dr. Fridtjof Nansen, 24. Mar, Oslo, Norway.
\emph{{[}Pecha Kucha{]} \textbf{Invited}}.

\leavevmode\hypertarget{ref-Langbehn1}{}%
\textbf{Langbehn, T.J.}, \& Varpe, Ø. (2015). \emph{Photoperiodic
implications on range expansion in polar marine ecosystems}. 6th
YOUMARES Conference, 16.- 18. Sep, Bremen, Germany. \emph{\textbf{Best
Speaker Award}}.

\endgroup

\hypertarget{posters}{%
\section{Posters}\label{posters}}

\begingroup
\setlength{\parindent}{-0.5in}
\setlength{\leftskip}{0.5in}

\hypertarget{refs_posters}{}
\leavevmode\hypertarget{ref-Langbehn13}{}%
\textbf{Langbehn, T.J.}, Bourgouin, A., Enberg, K., Hallfredsson, E. H.,
\& Jørgensen, C. (2021). \emph{Exploring the emergent niche of greater
argentine (argentina~silus) along gradients of topography, light and
advection}. ICES Annual Science Conference 2021, 6. - 9. Sep,
Copenhagen, Denmark. \emph{\textbf{accepted}}.

\leavevmode\hypertarget{ref-Langbehn12}{}%
\textbf{Langbehn, T.J.}, Jørgensen, C., \& Enberg, K. (2020).
\emph{Exploring the emergent niche of greater argentine
(argentina~silus) along gradients of topography, light and advection}.
Ocean Sciences Meeting 2020, 16.- 21. Feb, San Diego, USA.

\leavevmode\hypertarget{ref-Langbehn11}{}%
\textbf{Langbehn, T.J.}, Aarflot, J. M., \& Varpe, Ø. (2019).
\emph{Sea-ice loss is bad news for large copepods}. Nansen Legacy Annual
Meeting, 29.- 31. Oct, Oslo , Norway.

\leavevmode\hypertarget{ref-Langbehn10}{}%
\textbf{Langbehn, T.J.}, Aksnes, D. L., Kaartvedt, S., Fiksen, Ø., \&
Jørgensen, C. (2019). \emph{Poleward distribution of mesopelagic fish is
constrained by seasonality in light}. BIARRITZ (Bridging International
Activity and Related Research Into the Twilight Zone) workshop, 22.- 26.
Jun, Southampton, UK.

\leavevmode\hypertarget{ref-Langbehn9}{}%
\textbf{Langbehn, T.J.} (2017). \emph{Limits to northwards range shifts
of fish stocks under climate warming - a fish can't eat food it can't
find}. MARmaED midterm meeting, 11.- 12. Oct, Oslo , Norway.

\leavevmode\hypertarget{ref-Langbehn8}{}%
\textbf{Langbehn, T.J.}, Ljungström, G., Aksnes, D. L., Andersen, K. H.,
Olsen, Y., Ø., \& Jørgensen, C. (2017). \emph{Increasing marine food
production through ecolabeling}. Havforskermøtet, 20. Sep - 23. Nov,
Bergen, Norway.

\leavevmode\hypertarget{ref-Langbehn7}{}%
\textbf{Langbehn, T.J.}, Ljungström, G., Fiksen, Ø., \& Jørgensen, C.
(2017). \emph{Introducing a new hypothesis: Southward shifts of species
distributions under climate change?} 3rd International Workshop on
Trait-based Approaches to Ocean Life, 23.- 24. Mar, Bergen, Norway.

\leavevmode\hypertarget{ref-Langbehn3}{}%
\textbf{Langbehn, T.J.} (2016). \emph{Seasonal seas - limits to
northward range shifts of marine organisms under climate warming}.
MARmaED workshop, 28. Jul - 3. Aug, Banyuls sur mer, France.

\leavevmode\hypertarget{ref-Langbehn4}{}%
\textbf{Langbehn, T.J.}, \& Varpe, Ø. (2016). \emph{Photoperiodic
implications on visual foraging in polar marine ecosystems}. 16th
Congress of the International Society for Behavioural Ecology, 28. Jul -
3. Aug, Exeter, UK.

\leavevmode\hypertarget{ref-Langbehn5}{}%
\textbf{Langbehn, T.J.}, \& Varpe, Ø. (2016). \emph{Photoperiodic
implications on visual foraging in polar marine ecosystems}. 7th
YOUMARES Conference, 11.- 13. Sep, Hamburg, Germany. \emph{\textbf{Best
Poster Award}}.

\leavevmode\hypertarget{ref-Langbehn6}{}%
\textbf{Langbehn, T.J.}, \& Varpe, Ø. (2016). \emph{Photoperiodic
implications on visual foraging in polar marine ecosystems}. IMBER
ClimEco5 Summer School, 10.- 17. Aug, Natal, Brazil.
\emph{\textbf{Student Choice and Special Mention Poster Award}}.

\leavevmode\hypertarget{ref-Langbehn2}{}%
\textbf{Langbehn, T.J.}, \& Varpe, Ø. (2015). \emph{Visual search as a
trait: Consequences of sea ice change in high latitude oceans}. 2nd
International Workshop on Trait-based Approaches to Ocean Life, 5.- 8.
Oct, Waterville, New Hampshire, USA.

\leavevmode\hypertarget{ref-Langbehn1}{}%
\textbf{Langbehn, T.J.}, \& Landschoff, J. (2012). \emph{Live fast, die
young: What do changes in age structure tell us about management of cod
stocks?} Graduate Marine and Fisheries Sciences Summer Programme, Jun
2010, University of Iceland, Reykjavík, iceland. \emph{\textbf{Best
Student Poster}}.

\endgroup

\hypertarget{review-activities}{%
\section{Review activities}\label{review-activities}}

Reviewer for \emph{Global Change Biology}, \emph{Conservation
Physiology}, \emph{Marine Ecology Progress Series}, \emph{Ecological
Modelling}.

\end{document}
